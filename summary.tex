
\documentclass{article}[12pt]


\usepackage{amsfonts,amsmath,amscd,amssymb,amsthm}
\usepackage{graphicx}
\usepackage{geometry}
\usepackage{geometry}
\usepackage{setspace}
\usepackage{hyperref}
\geometry{ margin=1in}
\doublespacing


\title{\textbf{Summary on basic time series studies\\ \large tensor data analysis with different data types}}
\author{Haofan Zheng}
\date{}

\begin{document}


\maketitle
\newpage
\tableofcontents
\newpage

\section{High-dimentional $\alpha$-PCA method}
\subsection{Overall Summary}
This article considers the estimation and inference of the \textbf{low rank} components in high-dimentional matrixvariate models(tensor), and we propose an estimation method called $\alpha$-PCA and it has some benefits with the high dimensions data favorably compared with other methods(traditional PCA, etc) based on the perfomance in the simulation.



\subsection{Main model}
The model is shown as the following:

$$\mathbf{Y}_t = \mathbf{R}\mathbf{F}_t\mathbf{C}^T+\mathbf{E}_t$$

$\mathbf{Y_t}: \mathbf{Y_t}\in \mathbb{R}^{p\times q}$, $1\leq t \leq T$, observations,

$\mathbf{F_t}: \mathbf{F_t}\in \mathbb{R}^{k\times r}$, where $k\ll p$ and $r\ll q$ (\textbf{low rank}), latent matrix,

$\mathbf{E_t}: \mathbf{E_t}\in \mathbb{R}^{p \times q}$, noise matrix.

\subsection{Main Statistics}
An estimation procedure, namely $\alpha$-PCA, aggregates the information in both first and second moments. Specifically, the two statistics are defined:
$$\mathbf{\widehat{M}}_R  \overset{\Delta}{=} \dfrac{1}{pq}\Bigg((1+\alpha) \cdot \mathbf{\overline{Y}}\mathbf{\overline{Y}}^T+\dfrac{1}{T}\sum\limits_{t=1}^T(\mathbf{Y}_t-\mathbf{\overline{Y}})(\mathbf{Y}_t-\mathbf{\overline{Y}})^T\Bigg)$$

$$\mathbf{\widehat{M}}_C  \overset{\Delta}{=} \dfrac{1}{pq}\Bigg((1+\alpha) \cdot  \mathbf{\overline{Y}}^T\mathbf{\overline{Y}}+\dfrac{1}{T}\sum\limits_{t=1}^T(\mathbf{Y}_t-\mathbf{\overline{Y}})^T(\mathbf{Y}_t-\mathbf{\overline{Y}})\Bigg)$$

$\alpha: \alpha \in \left[-1,+\infty \right)$, a hyperparameter,

$\mathbf{\overline{Y}}=\dfrac{1}{T} \sum\limits_{i=1}^T\mathbf{Y}_t$, the sample mean.

Based on these two statistics, estimation of $\mathbf{R}$ and $\mathbf{C}$ can be obtained as $\sqrt{p}$ times the top $k$ eigenvectors of $\widehat{\mathbf{M}}_R$ and $\sqrt{q}$ times the top $q$ eigenvectors of $\widehat{\mathbf{M}}_C$ respectively, in descending order by corresponding eigenvalues.

\subsection{Theoretical Properties}

 

\subsection{Simulation}

\subsection{Application}
\section{High-Dimensional GLM with Binary Outcomes}
\subsection{Overall Summary}
\section [Ultra-High Dimensional GFM]{Ultra-High Dimensional GFM\footnote{Generalized Factor Model}} 
\subsection{Overall Summary} 

\section{Matrix-variate Logistic Regression with Measurement Error}
\section{A Likelihood-Based Approach for Multivariate Categorical Response Regression in High Dimensions}
\section{A likelihood-Based Approach for Semiparametric Regression with Panel Count Data}


\section{Time Series Latent Gaussian Count}
\section{Time Series Factor Models(tensor)}   

\end{document}
