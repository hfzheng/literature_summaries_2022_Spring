
\documentclass{article}[12pt]


\usepackage{amsfonts,amsmath,amscd,amssymb,amsthm}
\usepackage{graphicx}
\usepackage{geometry}
\usepackage{geometry}
\usepackage{setspace}
\geometry{ margin=1in}
\doublespacing


\title{\textbf{Summary on basic time series studies\\ \large tensor data analysis with different data types}}
\author{Haofan Zheng}
\date{}

\begin{document}


\maketitle
\newpage
\tableofcontents
\newpage

\section{$\alpha$-PCA method}
\subsection{Overall Summary}
This article considers the estimation and inference of the \textbf{low rank} components in high-dimentional matrixvariate models(tensor), and we propose an estimation method called $\alpha$-PCA and it has some benefits with the high dimensions data favorably compared with other methods(traditional PCA, etc) based on the perfomance in the simulation.



\subsection{Main model}
The model is shown as the following:

$$\mathbf{Y}_t = \mathbf{R}\mathbf{F}_t\mathbf{C}^T+\mathbf{E}_t$$

$\mathbf{Y_t}: \mathbf{Y_t}\in \mathbb{R}^{p\times q}$, $1\leq t \leq T$, observations

$\mathbf{F_t}: \mathbf{F_t}\in \mathbb{R}^{k\times r}$, where $k\ll p$ and $r\ll q$ (\textbf{low rank}), latent matrix

$\mathbf{E_t}: \mathbf{E_t}\in \mathbb{R}^{p \times q}$, noise matrix

\subsection{Main Statistics}

$$\mathbf{\hat{M}}_R  \overset{\Delta}{=} \dfrac{1}{pq}\Bigg((1+\alpha) \cdot \mathbf{}\Bigg)$$

$$\mathbf{\hat{M}}_C  \overset{\Delta}{=} \dfrac{1}{pq}\Bigg((1+\alpha) \cdot \Bigg)$$

$\alpha: \alpha \in \left[-1,+\infty \right)$, a hyperparameter 

$\mathbf{\bar{Y}}=\dfrac{1}{T} \sum\limits_{i=1}^T\mathbf{Y}_t$, the sample mean
\subsection{Theoretical Properties}
\subsection{Simulation}
\subsection{Application}
    
   

\end{document}
